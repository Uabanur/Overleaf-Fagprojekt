

This paper describes a somewhat successful attempt at making a silicon-based $2.8 \times 2.8$ \si{\micro m^2} optical broadband wavelength demultiplexer using an inverse design strategy based on topology optimisation. The design goal was to demultiplex an infrared broadband signal into two output signals peaking at 1300 nm and 1550 nm wavelengths, respectively. We obtained six different designs that, in finite-difference time-domain simulations, successfully demultiplexed the signal.
 All six designs had insertion losses (peak transmissions) around $-$0.5 dB and crosstalk around $-$17 dB. However, experiments on fabricated designs did not yield as satisfactory results. The best fabricated design had a peak insertion loss at $-$5 dB in the [1250, 1400] nm range, a peak insertion loss at $-$1 dB in the [1400, 1650] nm range, and peak crosstalk around $-$10 dB. The discrepancy is presumably attributed to a difficult translation from blueprint design to fabrication.


